% This document is part of the Imaging Archetypes project.
% Copyright 2014 the Authors.

% ## to-do
% - write

% ## style guidelines
% - {eqnarray} for equations!

\documentclass[12pt]{article}
\newcommand{\given}{\,|\,}
\begin{document}\sloppy\sloppypar

\noindent
\begin{tabular}{ll}
\textsl{title:}   & A translationally invariant dictionary model for imaging  \\
\textsl{from:}    & Hogg \\
\textsl{to:}      & Frean \\
\textsl{date:}    & 2014-05-03
\end{tabular}
\bigskip

It is not necessarily correct to think of a CCD pixel (or other
digital detector pixel) as a photon-collecting ``bucket''.
However, \emph{if} one does take this position, one can take the
telescope optics and atmospheric point-spread function and convolve
(or correlate) it with the pixel to get a pixel-convolved point-spread
function.
It is this object---the pixel-convolved point-spread function---that
we call the ``PSF'' from here on.
Because we have pre-convolved with the pixel, this PSF is just
\emph{sampled} at the pixel centers.

To be exceedingly specific, if the PSF is $\psi(\Delta x)$, where
$\Delta x$ is a focal-plane (two-dimensional) vector displacement
between pixel and star, then the expected level $\mu_m$ of pixel $m$
is
\begin{eqnarray}
\mu_m &=& I_0 + \sum_{k=1}^K S_k\,\psi(x_m - x_k)
\quad ,
\end{eqnarray}
where $I_0$ is some background or dc offset, there are $K$ stars $k$
with fluxes $S_k$ at two-dimensional vector positions $x_k$, and pixel
$m$ is at two-dimensional vector position $x_m$.
Here we have assumed that there are only stars, and they all have the
same PSF.
We have also implicitly assumed that all pixels are identical in their
size, shape, and mean sensitivity.
Note that because the imaging is not necessarily well-sampled, the
function $\psi(\Delta x)$ might need to be known or understood at a
resolution much finer than pixel-level.

If we have an image that contains $M$ measured pixel values (data
values) $d_m$ with inverse noise variances $w_m$ and the form given
above for the expected level, we can write down a likelihood that is
something like
\begin{eqnarray}
\ln p(d\given\theta) &=& -\frac{1}{2}\,\sum_{m=1}^M w_m\,[d_m - \mu_m(\theta)]^2 + \frac{1}{2}\,\sum_{m=1}^M \ln(\frac{w_m}{2\pi})
\\
p &\equiv& \left\{d_m\right\}_{m=1}^M
\\
\theta &\equiv& \left\{I_0, \left\{S_k, x_k\right\}_{k=1}^K\right\}
\\
\quad,
\end{eqnarray}
where the Gaussian assumption has been made, $d$ is a blob of data,
and $\theta$ is a blob of parameters.

If you want to be all cool about it, you can call this a ``dictionary
model'' for the $M$-pixel image.
This dictionary contains two ``words''.
The first word is the dc (uniform-illumination or sky) word, with
trivial ``shape''.
It appears only once in the image, with amplitude $I_0$.
The second word is the star word, with shape $\psi(\Delta x)$.
It appears $K$ times in the image, with amplitudes $S_k$ and
(continuous, real-valued, non-integer) offsets $x_k$.

\end{document}
