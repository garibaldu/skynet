% This document is part of the Imaging Archetypes project.
% Copyright 2014 the Authors.

% ## to-do
% - write

% ## style guidelines
% - {eqnarray} for equations!

\documentclass[12pt]{article}
\input{vc}
\newcommand{\given}{\,|\,}
\begin{document}\sloppy\sloppypar

\noindent
\begin{tabular}{ll}
\textsl{title:}    & A translationally invariant dictionary model for imaging \\
\textsl{from:}     & Hogg \\
\textsl{to:}       & Frean \\
\textsl{git hash:} & \texttt{\githash\,(\gitdate)}

\end{tabular}
\bigskip

It is not necessarily correct to think of a CCD pixel (or other
digital detector pixel) as a photon-collecting ``bucket''.
However, \emph{if} one does take this position, one can take the
telescope optics and atmospheric point-spread function and convolve
(or correlate) it with the pixel to get a pixel-convolved point-spread
function.
It is this object---the pixel-convolved point-spread function---that
we call the ``PSF'' from here on.
Because we have pre-convolved with the pixel, this PSF is just
\emph{sampled} at the pixel centers.

To be exceedingly specific, if the PSF is $\psi(\Delta x)$, where
$\Delta x$ is a focal-plane (two-dimensional) vector displacement
between pixel and star, then the expected level $\mu_m$ of pixel $m$
is
\begin{eqnarray}
\mu_m &=& I_0 + \sum_{k=1}^K s_k\,\psi(x_m - x_k) \label{eq:expect}
\quad ,
\end{eqnarray}
where $I_0$ is some background or dc offset, there are $K$ stars $k$
with fluxes $s_k$ at two-dimensional vector positions $x_k$, and pixel
$m$ is at two-dimensional vector position $x_m$.
Here we have assumed that there are only stars, and they all have the
same PSF.
We have also implicitly assumed that all pixels are identical in their
size, shape, and mean sensitivity.
Note that because the imaging is not necessarily well-sampled, the
function $\psi(\Delta x)$ might need to be known or understood at a
resolution much finer than pixel-level.

If we have an image that contains $M$ measured pixel values (data
values) $d_m$ with inverse noise variances $w_m$ and the form given
above in equation~(\ref{eq:expect}) for the expected level, we can
write down a likelihood that is something like
\begin{eqnarray}
\ln p(d\given\theta) &=& -\frac{1}{2}\,\sum_{m=1}^M w_m\,[d_m - \mu_m(\theta)]^2 + \frac{1}{2}\,\sum_{m=1}^M \ln(\frac{w_m}{2\pi}) \label{eq:like}
\\
d &\equiv& \left\{d_m\right\}_{m=1}^M
\\
\theta &\equiv& \left\{I_0, \left\{s_k, x_k\right\}_{k=1}^K\right\}
\quad,
\end{eqnarray}
where the Gaussian assumption has been made, $d$ is a blob of data,
and $\theta$ is a blob of parameters.

If you want to be all cool about it, you can call the description in
equation~(\ref{eq:expect}) a ``dictionary model'' for the $M$-pixel
image.
This dictionary contains two ``words''.
The first word is the dc (uniform-illumination or sky) word, with
trivial ``shape''.
It appears only once in the image, with amplitude $I_0$.
The second word is the star word, with shape $\psi(\Delta x)$.  It
appears $K$ times in the image, with amplitudes $s_k$ and (continuous,
real-valued, non-integer) two-dimensional positional offsets $x_k$.

Of course there are a million problems with the dictionary model
presented in equation~(\ref{eq:expect}).
The first is that it imagines that there are \emph{only} stars.
No galaxies or nebulae or cosmic rays!
The second is that it imagines that the PSF is both known and constant
over the image.
That is, the PSF model is rigid; it has neither parameters nor spatial
dependence.
Is it still a ``dictionary method'' if we soften these assumptions?
One softening would involve adding parameters $\theta_k$ to each
instance $k$ of the word.
Another softening would be to keep the words rigid (except for
amplitude $s_k$ and location $x_k$) but increase substantially the
\emph{number} of words in the vocabulary.

In the other direction, for computational simplicity, we can make the
model more rigid.
The first move in that direction would be to integerize or pixelize
the location space, so the word locations $x_k$ are also on an integer
grid like the pixel locations $x_m$.
Then instead of having $K$ words in the image, there would be $K$
pixels in the word-location space, with a prior that encourages most
of the $s_k$ to go to zero.
The transformation from the image of amplitudes $s_k$ to the image of
expectations (predictions) $\mu_m$ is in this case exactly a discrete
convolution.
There is fast shit for that.
Although I deliberately don't know much about fastness, I think that
it might even be fast if the image of amplitudes $s_k$ is far finer
(larger in dimension) than the image of data values $d_m$.
If I am right, we might be able to get the fastness of convolution and
\emph{also} the scientific value of super-resolution modeling.

When both the data space and the amplitude space are pixelized, the
expectation value becomes a finite linear algebra operation, which can
be written in two different ways.
\begin{eqnarray}
\mu &=& \Psi\,s \label{eq:linalg1}
\\
\mu &=& S\,\psi \label{eq:linalg2}
\quad ,
\end{eqnarray}
where $\mu$ is the $M$-vector of expectation values $\mu_m$ (unwrapped
from an image into a big vector), $\Psi$ is a $M\times K$ sparse
convolution matrix encoding the PSF $\psi(x_m - x_k)$, $s$ is the
$K$-vector of amplitudes $s_k$ (also unwrapped into a big vector), $S$
is a $M\times L$ sparse convolution matrix encoding the locations and
values of the non-zero amplitudes $s_k$, $\psi$ is an $L$-vector image
of the PSF at the fine resolution (again with the unwrapping), and I
have ignored the dc (sky) word.
These two expressions~(\ref{eq:linalg1}) and (\ref{eq:linalg2}) are
identical; they are just different ways of thinking about what's the
matrix and what's the vector.
I am being deliberately vague about the contents of the sparse
rectangular $\Psi$ and $S$ matrices, because the details depend a lot
on implementation (grids, unwrappings, and so on).
I have an intuition that there is an infinite-resolution
generalization of this way of writing it such that it would be
possible to work at infinite resolution without ever having to
represent in the computer any infinite dimensional objects, but I have
been drinking.
I also have an intuition that we will get maximum fastness only if the
$s_k$ pixel grid matches the $d_m$ pixel grid, but that's just because
I am uncreative.

The objective function would be something like
\begin{eqnarray}
-\frac{1}{2}\,||d - \mu||_2^2 - \frac{\epsilon}{2}\,||s||_1
\quad ,
\end{eqnarray}
where $\epsilon$ controls the strength of the regularization and maybe
we need to do some pixelwise multiplication of inverse-variance
weights in the real problem.
Better might be to write not an objective function but a likelihood
times a prior, and then do proper inference.

With a larger dictionary (that is, more than one word), notation
starts to become challenging...

\end{document}
